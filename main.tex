%%%%%%%%%%%%%%%%%%%%%%%%%%%%%%%%%%%%%%%%%%%%%%%%%%%%%%%%%%%%%%%%%%%%%%%%%%%%%%%%
%
%   IOIT ACM Event Report LaTeX Template
%   Version 1.0.1
%
%   Professional template for AISSMS IOIT ACM Student Chapter event reports.
%   This template provides a structured, clean layout for event reports including
%   workshops, competitions, seminars, hackathons, and other ACM activities.
%
%   Template: https://www.overleaf.com/read/nrdxngxxjqqp#a08899
%
%%%%%%%%%%%%%%%%%%%%%%%%%%%%%%%%%%%%%%%%%%%%%%%%%%%%%%%%%%%%%%%%%%%%%%%%%%%%%%%%
%
%   INSTRUCTIONS FOR AI LANGUAGE MODELS
%
%%%%%%%%%%%%%%%%%%%%%%%%%%%%%%%%%%%%%%%%%%%%%%%%%%%%%%%%%%%%%%%%%%%%%%%%%%%%%%%%
%
%   DOCUMENT STATE DETECTION
%
%   Before proceeding, analyze the document content to determine its state:
%
%   STATE 1 - Template Mode:
%   Indicators:
%   - Section bodies contain generic placeholder text in [brackets]
%   - Instructional comments explaining what to write are present
%   - Table cells contain example/placeholder data like "[Full Name of Resource Person]"
%   - Abstract/Introduction sections contain instructional text rather than actual content
%   - \newcommand definitions contain placeholder values like "DD/MM/YYYY" or "Event Name/Title Goes Here"
%
%   Action: Follow REPORT GENERATION PROTOCOL below.
%
%   STATE 2 - Completed Report Mode:
%   Indicators:
%   - Section bodies contain coherent, event-specific narrative prose
%   - Tables are populated with real data (actual names, dates, organizations)
%   - Abstract describes a specific event with concrete details
%   - \newcommand definitions contain actual event information
%   - No instructional comments remain in section bodies
%   - Content flows naturally and describes a real, completed event
%
%   Action: Follow REPORT UPDATE PROTOCOL below.
%
%   Decision Rule: If uncertain, treat as Template Mode. Read the Abstract and
%   Introduction sections - if they contain real event descriptions with specific
%   details, dates, and outcomes, it's a Completed Report. If they contain
%   instructions or generic placeholders, it's Template Mode.
%
%%%%%%%%%%%%%%%%%%%%%%%%%%%%%%%%%%%%%%%%%%%%%%%%%%%%%%%%%%%%%%%%%%%%%%%%%%%%%%%%
%
%   REPORT GENERATION PROTOCOL
%
%%%%%%%%%%%%%%%%%%%%%%%%%%%%%%%%%%%%%%%%%%%%%%%%%%%%%%%%%%%%%%%%%%%%%%%%%%%%%%%%
%
%   CRITICAL OUTPUT REQUIREMENT FOR GENERATED REPORTS:
%
%   When generating a complete report from template, you include the report update protocal
%   entire instruction block and mention all report update protocal update rules.
%   include the document structure rule as well. document structure rule should always
%   be retained as well without any changes.
%
%   This ensures that future modifications to the generated report can be handled
%   correctly using the UPDATE protocol.
%
%   RULE 1: DATA VERIFICATION REQUIREMENT
%
%   Do not generate the report without complete user-provided information.
%   If any required data is missing, stop and request it explicitly.
%
%   CRITICAL: When requesting missing data, output ONLY the list of missing
%   items. Do not include any other text, explanations, greetings, or comments.
%   Format: Present missing items as a simple categorized list with no additional
%   commentary.
%
%   RULE 2: REQUIRED DATA CHECKLIST
%
%   Universal Requirements (All Event Types):
%   - Event name, date (DD/MM/YYYY format), venue
%   - Event head name
%   - Report author name and submission date
%   - Total registrations and attendance numbers
%   - Event objectives and expected outcomes
%   - Detailed schedule with session timings
%   - Description of all activities conducted
%
%   Workshop/Seminar Specific:
%   - Resource person: Full name, designation, organization, email
%   - Resource person: Phone (optional), area of expertise, background
%   - Topics covered by resource person
%
%   Competition Specific:
%   - Winner details: Names or team names (1st, 2nd, 3rd place)
%   - Winner details: Department and year
%   - Prize information
%   - Judging criteria
%   - Problem statement or competition format
%
%   RULE 3: DATA COLLECTION PROCESS
%
%   When data is incomplete:
%   a. Identify all missing required fields
%   b. Present missing items as a categorized list
%   c. Request each piece of information explicitly
%   d. Output ONLY the missing data list - no other text
%   e. Wait for complete data before proceeding with generation
%
%   RULE 4: DOCUMENT GENERATION STANDARDS
%
%   Structure Requirements:
%   - Preserve all LaTeX structure: \documentclass, packages, custom commands
%   - Maintain all sections in prescribed order
%   - Keep \tableofcontents, \listoffigures, \listoftables
%   - Maintain all instructional comments from section bodies
%   - Remove all [bracketed placeholders]
%   - Retain only UPDATE protocol with all rules
%
%   Content Requirements:
%   - Replace all \newcommand definitions with actual user data
%   - Fill all table cells with real information
%   - Write professional, formal prose for descriptive sections
%   - Use image filenames: 1.jpeg, 2.jpeg, 3.jpeg, etc. (sequential numbering)
%   - Maintain consistent professional tone throughout
%   - NO EMOJIS anywhere in the generated report
%
%   Section Handling:
%   - Include "Resource Person Details" section for workshops/seminars/guest lectures
%   - Include "Competition Winners" section for competitions
%   - Comment out non-applicable sections with brief explanatory note
%
%   Output Validation:
%   - Final document must contain no [bracketed placeholders]
%   - Final document must contain no instructional comments except this header
%   - Final document must contain no emojis
%   - Document must be immediately compilable in LaTeX
%   - Document must be submission-ready except for image file replacements
%   - Image references must use sequential numbering: 1.jpeg, 2.jpeg, 3.jpeg
%   - This instruction block must be present at the top
%
%%%%%%%%%%%%%%%%%%%%%%%%%%%%%%%%%%%%%%%%%%%%%%%%%%%%%%%%%%%%%%%%%%%%%%%%%%%%%%%%
%
%   REPORT UPDATE PROTOCOL
%
%%%%%%%%%%%%%%%%%%%%%%%%%%%%%%%%%%%%%%%%%%%%%%%%%%%%%%%%%%%%%%%%%%%%%%%%%%%%%%%%
%
%   Maintain all header block instructions comments as they are, never change
%   or modify these and maintain all rules in comments.
%   Retain the document structure notice as well.
%
%   - Preserve all LaTeX structure: \documentclass, packages, custom commands
%   - Maintain all sections in prescribed order
%   - Keep \tableofcontents, \listoffigures, \listoftables
%
%   RULE 1: MINIMAL MODIFICATION PRINCIPLE
%
%   Apply surgical updates only. Modify exclusively the requested elements.
%   Preserve all other content exactly as written.
%
%   RULE 2: UPDATE SCOPE IDENTIFICATION
%
%   Determine precise modification targets:
%   - Metadata change: Modify only relevant \newcommand definitions
%   - Content change: Modify only the specific section or subsection
%   - Data change: Modify only relevant table cells or list items
%   - Addition: Insert new content without altering existing content
%
%   Examples:
%   - "Change event date to 15/03/2024" → Modify \newcommand{\reportdate} only
%   - "Add second prize winner" → Add single row to Competition Winners table
%   - "Update resource person email" → Modify email cell in Resource Person table
%   - "Expand outcomes section" → Modify Outcomes and Impact section only
%
%   RULE 3: PRESERVATION REQUIREMENTS
%
%   Do not modify:
%   - LaTeX structure, packages, or commands
%   - Sections not mentioned in update request
%   - Existing writing style, tone, or voice
%   - Working code or formatting
%   - Document organization or section order
%   - This instruction header block
%
%   RULE 4: UPDATE EXECUTION STANDARDS
%
%   Process:
%   a. Identify exact location of required change
%   b. Apply modification with minimal disruption
%   c. Maintain stylistic consistency with surrounding content
%   d. Verify document remains compilable
%   e. Ensure no emojis are introduced during updates
%   f. Return complete document from \documentclass to \end{document}
%   g. Preserve this instruction block at the top
%
%%%%%%%%%%%%%%%%%%%%%%%%%%%%%%%%%%%%%%%%%%%%%%%%%%%%%%%%%%%%%%%%%%%%%%%%%%%%%%%%
%
%   DOCUMENT STRUCTURE NOTICE
%
%   This template structure is institutionally mandated and immutable.
%
%   The following elements must never be modified:
%   - Document class and package declarations
%   - Custom command definitions structure
%   - Section order and required sections
%   - Front matter (table of contents, lists of figures/tables)
%   - Title page format and submission block
%
%   For updates: modify only user-requested content within sections.
%
%%%%%%%%%%%%%%%%%%%%%%%%%%%%%%%%%%%%%%%%%%%%%%%%%%%%%%%%%%%%%%%%%%%%%%%%%%%%%%%%

\documentclass[11pt, a4paper]{article}

%===============================================================================
%   1. PREAMBLE: PACKAGES AND BASIC SETUP
%===============================================================================

% --- Font and Encoding ---
\usepackage[utf8]{inputenc}
\usepackage[T1]{fontenc}

% --- Page Layout and Geometry ---
\usepackage[left=2.5cm, right=2.5cm, top=3.5cm, bottom=3cm, headheight=75pt]{geometry}

% --- Essential Packages ---
\usepackage{graphicx}
\usepackage{fancyhdr}
\usepackage{lastpage}
\usepackage{hyperref}
\usepackage{xcolor}

% --- Content and Formatting Packages ---
\usepackage{titlesec}
\usepackage{enumitem}
\usepackage{booktabs}
\usepackage{caption}
\usepackage{subcaption}
\usepackage{amsmath, amssymb}
\usepackage{url}
\usepackage{longtable}

%===============================================================================
%   2. DOCUMENT AND METADATA CONFIGURATION
%===============================================================================

% --- Hyperlink Setup ---
\hypersetup{
    colorlinks=true,
    linkcolor=blue,
    filecolor=magenta,
    urlcolor=cyan,
    pdftitle={ACM Event Report},
    pdfauthor={Your Name},
    pdfsubject={Event Report}
}

% --- Event Information (CUSTOMIZE THESE) ---
\newcommand{\reporttopic}{Event Name/Title Goes Here}
\newcommand{\reportdate}{DD/MM/YYYY}
\newcommand{\reportvenue}{AISSMS Institute of Information Technology}
\newcommand{\reportorganizer}{AISSMS IOIT ACM Student Chapter}
\newcommand{\eventhead}{Name of Event head}
\newcommand{\reportcollaboration}{Collaboration Partner (if any)}
\newcommand{\reportauthor}{Report Author Name}
\newcommand{\submissiondate}{DD/MM/YYYY}

%===============================================================================
%   3. HEADER AND FOOTER CONFIGURATION (using fancyhdr)
%===============================================================================

\pagestyle{fancy}
\fancyhf{}

% --- Header Definition for Main Pages (Page 2 onwards) ---
\fancyhead[L]{\raisebox{0.3cm}{\includegraphics[height=1.5cm]{acm-logo.png}}}
\fancyhead[C]{\raisebox{0.2cm}{\includegraphics[height=2cm]{college-logo.png}}}
\renewcommand{\headrulewidth}{0.4pt}

% --- Footer Definition ---
\fancyfoot[C]{\thepage}
\renewcommand{\footrulewidth}{0pt}

%===============================================================================
%   4. CUSTOM COMMANDS
%===============================================================================

% --- Custom Title Page Command ---
\newcommand{\makecustomtitle}{
\begin{titlepage}
\centering
\thispagestyle{empty}

\noindent
\makebox[0.32\textwidth][l]{%
    \includegraphics[height=2.4cm]{acm-logo.png}
}%
\makebox[0.36\textwidth][c]{%
    \includegraphics[height=2.3cm]{college-logo.png}
}%
\makebox[0.32\textwidth][r]{%
}
\vspace{2.5cm}

{\Large \textbf{AISSMS IOIT ACM Student Chapter Event Report}}\\[0.5cm]
{\Huge \bfseries \reporttopic}\\
\vfill

\begin{flushleft}
\large
\begin{tabular}{@{}l@{\quad}l}
\textbf{EVENT DATE:} & \reportdate \\
\textbf{VENUE:} & \reportvenue \\
\textbf{ORGANIZED BY:} & \reportorganizer \\
\textbf{Event Head:} & \eventhead \\
% \textbf{IN COLLABORATION WITH:} & \reportcollaboration \\ % Uncomment if applicable
\end{tabular}
\end{flushleft}

\vfill
\end{titlepage}
}


%===============================================================================
%   5. DOCUMENT BODY BEGINS
%===============================================================================
\begin{document}

% --- Title Page ---
\makecustomtitle

% --- Front Matter (Table of Contents, etc.) ---
\pagenumbering{roman}
\newpage
\tableofcontents
\listoffigures
\listoftables
\newpage

% --- Main Content ---
\pagenumbering{arabic}

%-------------------------------------------------------------------------------
\section{Abstract}
% PURPOSE: This section provides a concise summary of the entire event (150-250 words).
% CONTENT TO INCLUDE:
% - Brief overview of the event and its purpose
% - Event type (workshop, competition, seminar, hackathon, etc.)
% - Main objectives of organizing this event
% - Key activities conducted
% - Number of participants/attendees
% - Primary outcomes or key highlights
% Keep it brief and self-contained - readers should understand the essence of
% the event without reading the full report.

[Write your abstract here - a concise summary of the event, its objectives, activities conducted, participation, and key outcomes in 150-250 words]

%-------------------------------------------------------------------------------
\section{Introduction}
% PURPOSE: Set the context and background for the event.
% CONTENT TO INCLUDE:
% - Background about why this event was organized
% - Relevance to ACM Student Chapter's mission/goals
% - Target audience (students, professionals, specific departments, etc.)
% - Current trends or needs in the domain that this event addresses
% - Brief overview of what the report will cover
% - Any historical context (if this is a recurring event)
% This section should gradually lead the reader from general context to
% the specific event.

[Write your introduction here - provide background context, explain why this event was organized, its relevance to students/community, and give an overview of the report]

%-------------------------------------------------------------------------------
\section{Event Objective}
% PURPOSE: Clearly state what the event aimed to achieve.
% CONTENT TO INCLUDE:
% - Primary goal(s) of organizing the event
% - Learning outcomes expected for participants
% - Skills or knowledge to be developed
% - Community building or networking objectives
% - Expected impact on students/participants
% - Specific deliverables or outcomes planned
% Be specific and measurable where possible.

[State the main goals and specific objectives of the event - what did you aim to achieve? What learning outcomes were expected? What impact did you plan to create?]

%-------------------------------------------------------------------------------
\section{Event Details and Structure}
% PURPOSE: Describe the organization and structure of the event.
% CONTENT TO INCLUDE:
% - Event format (workshop, competition, seminar, panel discussion, etc.)
% - Duration and schedule (single-day, multi-day, time slots)
% - Registration process and participant selection (if applicable)
% - Event agenda or timeline
% - Activities/sessions conducted
% - Tools, platforms, or technologies used (online/offline)
% - Team involved in organizing (roles and responsibilities)
% This section should give readers a clear understanding of how the
% event was structured and executed.

[Describe the format, structure, schedule, and organization of the event. Include the agenda, activities conducted, and how the event was managed]

\subsection{Event Agenda}
% PURPOSE: Provide detailed schedule of the event.
% CONTENT TO INCLUDE:
% - Session-wise breakdown with timings
% - Topics covered in each session
% - Speaker/facilitator for each session
% - Break times and networking sessions
% Use a table or itemized list for clarity.

[Provide the detailed agenda/schedule of the event with timings and session details]

\begin{table}[h]
    \centering
    \caption{Event Schedule}
    \label{tab:schedule}
    \begin{tabular}{@{}lll@{}}
        \toprule
        \textbf{Time} & \textbf{Session/Activity} \\
        \midrule
        10:00 AM - 10:30 AM & Registration \& Welcome \\
        10:30 AM - 11:30 AM & [Session 1 Topic] \\
        11:30 AM - 12:30 PM & [Session 2 Topic] \\
        12:30 PM - 01:30 PM & Lunch Break & \\
        01:30 PM - 03:00 PM & [Session 3 Topic] \\
        03:00 PM - 04:00 PM & [Final Activity/Closing] \\
        \bottomrule
    \end{tabular}
\end{table}

%-------------------------------------------------------------------------------
% DYNAMIC SECTION 1: RESOURCE PERSON DETAILS
% INSTRUCTIONS: Keep this section for WORKSHOPS, SEMINARS, GUEST LECTURES, and
% similar events where external experts/speakers are invited.
% COMMENT OUT this entire section for events like COMPETITIONS, HACKATHONS
% (unless they have guest speakers/judges), or INTERNAL ACTIVITIES.
%-------------------------------------------------------------------------------
\section{Resource Person Details}
% PURPOSE: Provide information about the expert(s) who conducted the event.
% CONTENT TO INCLUDE:
% - Complete details as per the table format
% - Name, designation, organization/company
% - Email ID and contact information
% - Area of expertise relevant to the event
% - Brief background or credentials (can be added in paragraph form below table)
% - Topics/sessions they handled
% This section is crucial for workshops, seminars, and guest lectures.
% COMMENT OUT THIS SECTION if your event type doesn't have resource persons.

[Fill in the details of the resource person(s) who conducted the workshop/seminar. Use the table format provided below]

\begin{table}[h]
    \centering
    \caption{Details of Industry Expert / Resource Person}
    \label{tab:resource_person}
    \begin{tabular}{|p{0.35\textwidth}|p{0.60\textwidth}|}
        \hline
        \textbf{Name} & [Full Name of Resource Person] \\
        \hline
        \textbf{Designation} & [Job Title/Position] \\
        \hline
        \textbf{Organization/Company} & [Organization/Company Name] \\
        \hline
        \textbf{Email ID} & [email@example.com] \\
        \hline
        \textbf{Area of Expertise} & [e.g., Machine Learning, Web Development, Cloud Computing, Cybersecurity, etc.] \\
        \hline
    \end{tabular}
\end{table}

% If multiple resource persons, duplicate the table:
% \begin{table}[h]
%     \centering
%     \caption{Details of Resource Person 2}
%     \label{tab:resource_person2}
%     [Same table structure as above]
% \end{table}

%-------------------------------------------------------------------------------
% DYNAMIC SECTION 2: COMPETITION WINNERS
% INSTRUCTIONS: Keep this section for COMPETITIONS, HACKATHONS, CODING CONTESTS,
% and similar competitive events.
% COMMENT OUT this entire section for events like WORKSHOPS, SEMINARS, GUEST
% LECTURES, or NON-COMPETITIVE ACTIVITIES.
%-------------------------------------------------------------------------------
\section{Competition Winners}
% PURPOSE: Recognize and document the winners of the competition.
% CONTENT TO INCLUDE:
% - Winner details with positions (1st, 2nd, 3rd place, special mentions)
% - Participant names (individual or team names)
% - Department/Year (if applicable)
% - Brief description of winning entry/performance
% - Prizes awarded
% This section is essential for all competitive events.
% COMMENT OUT THIS SECTION if your event is not a competition.

[Document the competition winners with their positions, names, and details. Add photos if available]

\subsection{Winner List}
% List the winners in order of their rankings

\begin{table}[h]
    \centering
    \caption{Competition Winners}
    \label{tab:winners}
    \begin{tabular}{|c|p{0.30\textwidth}|p{0.25\textwidth}|p{0.25\textwidth}|}
        \hline
        \textbf{Position} & \textbf{Name/Team Name} & \textbf{Department/Year} \\
        \hline
        1st Place & [Winner Name/Team] & [Dept/Year] \\
        \hline
        2nd Place & [Winner Name/Team] & [Dept/Year] \\
        \hline
        3rd Place & [Winner Name/Team] & [Dept/Year] \\
        \hline
        % Add more rows for special mentions, consolation prizes, etc.
    \end{tabular}
\end{table}

%-------------------------------------------------------------------------------
\section{Event Execution and Activities}
% PURPOSE: Describe what actually happened during the event.
% CONTENT TO INCLUDE:
% - Detailed description of sessions/activities conducted
% - Key topics covered or skills taught
% - Hands-on activities, exercises, or tasks
% - Participant engagement and interaction
% - Demonstrations, presentations, or practical sessions
% - Q&A sessions, discussions, or collaborative activities
% This section should give readers a vivid picture of the event experience.

[Describe the actual execution of the event - what activities were conducted, what topics were covered, how participants engaged, and any notable moments]

%-------------------------------------------------------------------------------
\section{Participation and Engagement}
% PURPOSE: Document participant statistics and engagement levels.
% CONTENT TO INCLUDE:
% - Total number of participants/registrations
% - Department-wise or year-wise breakdown
% - Attendance statistics
% - Participant feedback or response during the event
% - Level of engagement and interaction
% - Any participation certificates or recognition given
% Use tables or charts to present statistics clearly.

[Provide detailed information about participation numbers, demographics, and engagement levels during the event]

\begin{table}[h]
    \centering
    \caption{Participation Statistics}
    \label{tab:participation}
    \begin{tabular}{@{}lc@{}}
        \toprule
        \textbf{Category} & \textbf{Count} \\
        \midrule
        Total Registrations & [Number] \\
        Total Attendees & [Number] \\
        \bottomrule
    \end{tabular}
\end{table}

%-------------------------------------------------------------------------------
\section{Outcomes and Impact}
% PURPOSE: Present and analyze the results and impact of the event.
% CONTENT TO INCLUDE:
% - Whether event objectives were achieved
% - Learning outcomes for participants
% - Skills developed or knowledge gained
% - Feedback received from participants
% - Unexpected positive outcomes or learnings
% - Success metrics (attendance vs. registration, engagement levels, etc.)
% This section should demonstrate the value and impact of organizing the event.

[Describe the outcomes and impact of the event. How successful was it? What did participants learn or gain? Was the event effective in achieving its objectives?]

\subsection{Participant Feedback}
% PURPOSE: Include qualitative feedback from attendees.
% CONTENT TO INCLUDE:
% - Summary of feedback collected (surveys, forms, verbal feedback)
% - Positive comments and testimonials
% - Areas of improvement suggested by participants
% - Overall satisfaction ratings
% - Quotes from participants (if available)

[Summarize the feedback received from participants. Include both quantitative ratings and qualitative comments]

\begin{itemize}
    \item Overall satisfaction rating: [X/5 or percentage]
    \item Positive feedback highlights: [Key positive comments]
    \item Areas for improvement: [Suggestions from participants]
    \item Notable testimonials: [Direct quotes if available]
\end{itemize}

\subsection{Knowledge and Skills Gained}
% PURPOSE: Document the educational value of the event.
% CONTENT TO INCLUDE:
% - Technical skills acquired or improved
% - Soft skills developed (teamwork, communication, problem-solving)
% - Practical exposure to industry tools/technologies
% - Career or academic insights gained
% - Networking opportunities created

[Describe the knowledge, skills, and experience gained by participants through this event. What specific competencies were developed?]

\begin{itemize}
    \item Technical skill 1: [Description]
    \item Technical skill 2: [Description]
    \item Soft skill 1: [Description]
    \item Practical exposure: [Description]
    \item Career insights: [Description]
\end{itemize}

\subsection{Long-term Impact}
% PURPOSE: Discuss the broader or lasting impact of the event.
% CONTENT TO INCLUDE:
% - How this event contributes to students' career readiness
% - Community building within the ACM chapter
% - Industry connections established
% - Follow-up activities or continued learning opportunities

[Discuss the long-term or broader impact of this event on participants, the ACM chapter, and the student community]

%-------------------------------------------------------------------------------
\section{Glimpses from the Event}
% PURPOSE: Provide visual documentation of the event.
% CONTENT TO INCLUDE:
% - Photos from the event (venue, participants, activities, presentations)
% - Screenshots (for online events or demonstrations)
% - Resource person delivering sessions
% - Participant engagement moments
% - Group photos
% Each figure should have a clear, descriptive caption explaining what it shows.
% Reference figures in the text using \ref{fig:label}
% Image files should be named: 1.jpeg, 2.jpeg, 3.jpeg, etc.

[Include photographs and other visual documentation from the event. Each figure should have a descriptive caption. Replace image placeholders with actual filenames: 1.jpeg, 2.jpeg, 3.jpeg]

\begin{figure}[h]
    \centering
    \includegraphics[width=0.8\textwidth]{1.jpeg}
    \caption{[Caption describing the image - e.g., "Resource person delivering the workshop session on Machine Learning"]}
    \label{fig:session}
\end{figure}

\begin{figure}[h]
    \centering
    \begin{subfigure}{0.48\textwidth}
        \includegraphics[width=\linewidth]{2.jpeg}
        \caption{[Caption for first image]}
        \label{fig:glimpse1}
    \end{subfigure}
    \hfill
    \begin{subfigure}{0.48\textwidth}
        \includegraphics[width=\linewidth]{3.jpeg}
        \caption{[Caption for second image]}
        \label{fig:glimpse2}
    \end{subfigure}
    \caption{[Overall caption - e.g., "Participants actively engaged in hands-on activities"]}
    \label{fig:activities}
\end{figure}

%-------------------------------------------------------------------------------
\section{Conclusion}
% PURPOSE: Wrap up the report and provide recommendations.
% CONTENT TO INCLUDE:
% - Summary of the event's success
% - Restatement of how objectives were met
% - Overall impact on participants and ACM chapter
% - Potential improvements or enhancements
% - Suggestions for follow-up activities or next steps
% - Vision for continuing this event (if recurring)
% Keep this section concise and forward-looking.

[Summarize the event's success, restate achievement of objectives]

%-------------------------------------------------------------------------------
\section*{Acknowledgements}
% PURPOSE: Recognize contributions and support.
% CONTENT TO INCLUDE:
% - Faculty coordinator (Dr. Meenakshi Thalor) and mentors
% - ACM chapter office bearers and organizing committee members
% - Resource persons/guest speakers/judges
% - Collaborating organizations, companies, or industry partners
% - Sponsors or funding sources (if applicable)
% - College administration and departments for support
% - Volunteers and supporting staff
% - Any other individuals or groups who contributed
% Be specific and sincere in expressing gratitude.

[Acknowledge all the individuals, faculty members, organizations, and institutions that supported and contributed to making this event successful. Express gratitude for their guidance, resources, and collaboration]

We would like to express our sincere gratitude to:
\begin{itemize}
    \item Faculty Coordinator Dr. Meenakshi Thalor, for guidance and support
    \item Resource Person Name, for conducting an excellent workshop/seminar/session
    \item College Department, for providing venue and resources
    \item Sponsoring Organization, for their generous support
    \item ACM Student Chapter office bearers and organizing team members
    \item All participants for their enthusiastic engagement
\end{itemize}

%===============================================================================
%   SUBMISSION BLOCK
%===============================================================================
\newpage
\vfill
\begin{flushleft}
    \large
    \textbf{REPORT SUBMITTED BY:} \\
    \vspace{0.5cm}
    \reportauthor \\
    AISSMS IOIT ACM Student Chapter \\
    \vspace{2cm}
    \rule{8cm}{0.4pt} \\
    (Signature) \\
    \vspace{1cm}
    \textbf{Date:} \submissiondate
\end{flushleft}


\end{document}
%===============================================================================
%   END OF DOCUMENT
%===============================================================================
