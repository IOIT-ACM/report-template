%%%%%%%%%%%%%%%%%%%%%%%%%%%%%%%%%%%%%%%%%%%%%%%%%%%%%%%%%%%%%%%%%%%%%%%%%%%%%%%%
%
%   IOIT ACM Event Report LaTeX Template
%
%   Professional template for AISSMS IOIT ACM Student Chapter event reports.
%   Supports both standalone and annual report compilation modes.
%   Version 2.1.0
%
%%%%%%%%%%%%%%%%%%%%%%%%%%%%%%%%%%%%%%%%%%%%%%%%%%%%%%%%%%%%%%%%%%%%%%%%%%%%%%%%
%
%   INSTRUCTION HEADER — READ BEFORE EDITING
%
%%%%%%%%%%%%%%%%%%%%%%%%%%%%%%%%%%%%%%%%%%%%%%%%%%%%%%%%%%%%%%%%%%%%%%%%%%%%%%%%
%
%   NON-NEGOTIABLE RULES
%
%   The following elements are institutionally mandated and MUST NEVER be modified:
%   - Document class and package declarations
%   - Custom command definitions structure
%   - Section order and required sections
%   - This instruction header
%
%   Violation of these rules invalidates the report for institutional submission.
%
%%%%%%%%%%%%%%%%%%%%%%%%%%%%%%%%%%%%%%%%%%%%%%%%%%%%%%%%%%%%%%%%%%%%%%%%%%%%%%%%
%
%   GENERATION PROTOCOL
%
%%%%%%%%%%%%%%%%%%%%%%%%%%%%%%%%%%%%%%%%%%%%%%%%%%%%%%%%%%%%%%%%%%%%%%%%%%%%%%%%
%
%   RULE 1: DATA VERIFICATION REQUIREMENT
%
%   Do not generate without complete user-provided information.
%   If data is missing, output ONLY a categorized list of missing items.
%   No greetings, explanations, or commentary—just the list.
%
%   RULE 2: REQUIRED DATA CHECKLIST
%
%   Universal (All Events):
%   - Event name, date (DD/MM/YYYY), venue
%   - Event head name, report author name, submission date
%   - Registration and attendance numbers
%   - Event objectives, expected outcomes, detailed schedule
%   - Description of all activities conducted
%
%   Workshop/Seminar Specific:
%   - Resource person: Full name, designation, organization, email
%   - Resource person: Expertise area, background, topics covered
%
%   Competition Specific:
%   - Winner details: Names/team names (1st, 2nd, 3rd)
%   - Winner details: Department, year, prize information
%   - Judging criteria, problem statement/format
%
%   RULE 3: GENERATION STANDARDS
%
%   Structure:
%   - Preserve all LaTeX structure elements
%   - Maintain prescribed section order
%   - Remove all [bracketed placeholders]
%   - Keep ONLY this instruction header and Update Protocol
%
%   Content:
%   - Replace \renewcommand definitions with actual data
%   - Write professional, formal prose
%   - Use sequential image filenames: 1.jpeg, 2.jpeg, 3.jpeg
%   - NO EMOJIS anywhere
%   - Include/comment out sections based on event type
%   - Document must be immediately compilable and submission-ready
%
%%%%%%%%%%%%%%%%%%%%%%%%%%%%%%%%%%%%%%%%%%%%%%%%%%%%%%%%%%%%%%%%%%%%%%%%%%%%%%%%
%
%   UPDATE PROTOCOL
%
%%%%%%%%%%%%%%%%%%%%%%%%%%%%%%%%%%%%%%%%%%%%%%%%%%%%%%%%%%%%%%%%%%%%%%%%%%%%%%%%
%
%   RULE 1: MINIMAL MODIFICATION PRINCIPLE
%
%   Apply surgical updates only. Modify exclusively requested elements.
%   Preserve all other content exactly as written.
%
%   RULE 2: UPDATE SCOPE IDENTIFICATION
%
%   Examples:
%   - "Change event date to 15/03/2024" → Modify \renewcommand{\reportdate} only
%   - "Add second prize winner" → Add single row to winners table only
%   - "Update resource person email" → Modify email cell only
%   - "Expand outcomes section" → Modify Outcomes section only
%
%   RULE 3: PRESERVATION REQUIREMENTS
%
%   Never modify:
%   - LaTeX structure, packages, commands (per Non-Negotiable Rules)
%   - Sections not mentioned in update request
%   - Existing writing style, tone, or voice
%   - This instruction header
%
%   RULE 4: EXECUTION STANDARDS
%
%   Process:
%   a. Identify exact location of required change
%   b. Apply modification with minimal disruption
%   c. Maintain stylistic consistency
%   d. Verify document remains compilable
%   e. Ensure no emojis introduced
%   f. Return complete document with this header preserved
%   2. The content starts after the Context-Aware Wrapper.
%   3. If this event is NOT a competition, comment out the "Competition Winners" section.
%   4. If this event is NOT a workshop/seminar, comment out "Resource Person Details".
%
%%%%%%%%%%%%%%%%%%%%%%%%%%%%%%%%%%%%%%%%%%%%%%%%%%%%%%%%%%%%%%%%%%%%%%%%%%%%%%%%
%
%   LATEX COMPILATION SAFETY RULES (MANDATORY)
%
%%%%%%%%%%%%%%%%%%%%%%%%%%%%%%%%%%%%%%%%%%%%%%%%%%%%%%%%%%%%%%%%%%%%%%%%%%%%%%%%
%
%   The generated document MUST compile without errors or warnings.
%   The following rules are strictly enforced during content generation.
%
%%%%%%%%%%%%%%%%%%%%%%%%%%%%%%%%%%%%%%%%%%%%%%%%%%%%%%%%%%%%%%%%%%%%%%%%%%%%%%%%
%
%   TABLE AND ALIGNMENT SAFETY
%
%   1. SPECIAL CHARACTER ESCAPING
%      The following characters MUST be escaped when used as text:
%
%         &   → \&
%         %   → \%
%         _   → \_
%         #   → \#
%         $   → \$
%
%      These characters are reserved by LaTeX and MUST NEVER appear
%      unescaped inside tables, captions, or normal text.
%
%   2. COLUMN ALIGNMENT RULE
%      The number of '&' separators in each table row MUST EXACTLY match
%      the column definition of the tabular/tabularx environment.
%
%      Example:
%          {lX}  → exactly ONE '&' per row.
%          {cXX} → exactly TWO '&' per row.
%
%      Extra or missing '&' characters are strictly prohibited.
%
%   3. ROW TERMINATION
%      Every table row MUST end with:
%
%          \\
%
%      before any \hline, \midrule, or \bottomrule command.
%
%   4. LONG TEXT IN TABLE CELLS
%      Long paragraphs inside tables MUST be wrapped using:
%
%          \parbox[t]{\linewidth}{...}
%
%      or allowed to flow only within X-type columns of tabularx.
%      Plain long text must not overflow alignment.
%
%   5. NO EMPTY TABLE ROWS
%      Blank rows or incomplete rows inside tables are not allowed.
%
%%%%%%%%%%%%%%%%%%%%%%%%%%%%%%%%%%%%%%%%%%%%%%%%%%%%%%%%%%%%%%%%%%%%%%%%%%%%%%%%
%
%   BOOKTABS USAGE RULES
%
%   6. When using \toprule, \midrule, or \bottomrule:
%         - Do NOT combine them with vertical lines.
%         - Do NOT place text immediately after rule commands.
%
%%%%%%%%%%%%%%%%%%%%%%%%%%%%%%%%%%%%%%%%%%%%%%%%%%%%%%%%%%%%%%%%%%%%%%%%%%%%%%%%
%
%   ENVIRONMENT INTEGRITY
%
%   7. Every environment MUST close properly:
%
%          \begin{tabularx} → \end{tabularx}
%          \begin{table}    → \end{table}
%          \begin{figure}   → \end{figure}
%
%      Nested environments must close in reverse order.
%
%%%%%%%%%%%%%%%%%%%%%%%%%%%%%%%%%%%%%%%%%%%%%%%%%%%%%%%%%%%%%%%%%%%%%%%%%%%%%%%%
%
%   CONTENT SAFETY RULES
%
%   8. Do NOT introduce additional LaTeX packages.
%   9. Do NOT modify column definitions.
%  10. Do NOT insert raw line breaks or manual spacing commands inside tables.
%  11. Do NOT include emojis or Unicode symbols unsupported by LaTeX.
%
%%%%%%%%%%%%%%%%%%%%%%%%%%%%%%%%%%%%%%%%%%%%%%%%%%%%%%%%%%%%%%%%%%%%%%%%%%%%%%%%
%
%   VALIDATION REQUIREMENT
%
%   12. Before output, the document must be internally validated to ensure:
%       - No alignment mismatches exist.
%       - No unescaped special characters remain.
%       - All tables compile cleanly.
%
%%%%%%%%%%%%%%%%%%%%%%%%%%%%%%%%%%%%%%%%%%%%%%%%%%%%%%%%%%%%%%%%%%%%%%%%%%%%%%%%

%===============================================================================
%   CONTEXT-AWARE WRAPPER
%===============================================================================
\ifdefined\annualreportmode
    % Mode: Annual Report - Preamble is already loaded by the master file.
\else
    % Mode: Standalone - Load essential configuration.
    \documentclass[11pt, a4paper]{article}
    %%%%%%%%%%%%%%%%%%%%%%%%%%%%%%%%%%%%%%%%%%%%%%%%%%%%%%%%%%%%%%%%%%%%%%%%%%%%%%%%
%
%   COMMON PREAMBLE FOR IOIT ACM EVENT REPORTS
%   This file contains all shared package declarations and configurations
%   Used by both standalone event reports and the annual report
%
%%%%%%%%%%%%%%%%%%%%%%%%%%%%%%%%%%%%%%%%%%%%%%%%%%%%%%%%%%%%%%%%%%%%%%%%%%%%%%%%

% --- Font and Encoding ---
\usepackage[utf8]{inputenc}
\usepackage[T1]{fontenc}

% --- Page Layout and Geometry ---
\usepackage[left=2cm, right=2cm, top=3.5cm, bottom=2.5cm, headheight=75pt]{geometry}

% --- Essential Packages ---
\usepackage{graphicx}
\usepackage{fancyhdr}
\usepackage{lastpage}
\usepackage{hyperref}
\usepackage{xcolor}

% --- Content and Formatting Packages ---
\usepackage{titlesec}
\usepackage{enumitem}
\usepackage{booktabs}
\usepackage{caption}
\usepackage{subcaption}
\usepackage{amsmath, amssymb}
\usepackage{url}
\usepackage{longtable}
\usepackage{tabularx}

% --- Hyperlink Setup ---
\hypersetup{
    colorlinks=true,
    linkcolor=blue,
    filecolor=magenta,
    urlcolor=cyan,
    pdftitle={ACM Event Report},
    pdfauthor={AISSMS IOIT ACM Student Chapter},
    pdfsubject={Event Report}
}

% --- Header and Footer Configuration ---
\pagestyle{fancy}
\fancyhf{}

% Header Definition
\fancyhead[L]{\raisebox{0.3cm}{\includegraphics[height=1.5cm]{acm-logo.png}}}
\fancyhead[C]{\raisebox{0.2cm}{\includegraphics[height=2cm]{college-logo.png}}}
\renewcommand{\headrulewidth}{0.4pt}

% Footer Definition
\fancyfoot[C]{\thepage}
\renewcommand{\footrulewidth}{0pt}

% no header style
\fancypagestyle{noheader}{
\fancyhf{}
\fancyfoot[C]{\thepage}
\renewcommand{\headrulewidth}{0pt}
\renewcommand{\footrulewidth}{0pt}
}

    %%%%%%%%%%%%%%%%%%%%%%%%%%%%%%%%%%%%%%%%%%%%%%%%%%%%%%%%%%%%%%%%%%%%%%%%%%%%%%%%
%
%   EVENT REPORT MACROS
%   Updated for Arabic Pagination and Conditional Front Matter
%
%%%%%%%%%%%%%%%%%%%%%%%%%%%%%%%%%%%%%%%%%%%%%%%%%%%%%%%%%%%%%%%%%%%%%%%%%%%%%%%%

% --- Mode Detection ---
\newif\ifstandalone
\ifdefined\annualreportmode
    \standalonefalse
\else
    \standalonetrue
\fi

% --- Metadata Placeholder Definitions ---
% These are default values. Reports will override these using \renewcommand.
\renewcommand{\reporttopic}{Event Name/Title Goes Here}
\renewcommand{\reportdate}{DD/MM/YYYY}
\renewcommand{\reportvenue}{AISSMS Institute of Information Technology}
\renewcommand{\reportorganizer}{AISSMS IOIT ACM Student Chapter}
\renewcommand{\eventhead}{Name of Event head}
\renewcommand{\reportauthor}{Report Author Name}
\renewcommand{\submissiondate}{DD/MM/YYYY}
\renewcommand{\reportcollaboration}{} % Add if needed

% --- Section Reset Logic for Annual Report ---
\newcounter{annualtempsection}
\newcounter{eventcount}

\newcommand{\includeevent}[1]{%
    \stepcounter{eventcount}%
    \setcounter{annualtempsection}{\value{section}}
    \setcounter{section}{0}
    \renewcommand{\theHsection}{event.\theeventcount.\arabic{section}}
    \input{#1}
    \setcounter{section}{\value{annualtempsection}}
    \renewcommand{\theHsection}{\arabic{section}}
}

% --- Custom Title Page Command ---
\newcommand{\makecustomtitle}{%
\ifstandalone
    \begin{titlepage}
    \thispagestyle{empty}
\else
    \clearpage
    \thispagestyle{fancy} % Standard header for annual report pages
\fi

    \centering
    \noindent
    \makebox[0.32\textwidth][l]{\includegraphics[height=2.4cm]{acm-logo.png}}%
    \makebox[0.36\textwidth][c]{\includegraphics[height=2.3cm]{college-logo.png}}%
    \makebox[0.32\textwidth][r]{}
    \vspace{2.5cm}

    {\Large \textbf{AISSMS IOIT ACM Student Chapter Event Report}}\\[0.5cm]
    {\Huge \bfseries \reporttopic}\\
    \vfill

    \begin{flushleft}
    \large
    \begin{tabular}{@{}l@{\quad}l}
    \textbf{EVENT DATE:} & \reportdate \\
    \textbf{VENUE:} & \reportvenue \\
    \textbf{ORGANIZED BY:} & \reportorganizer \\
    \ifx\reportcollaboration\empty\else
    \textbf{IN COLLABORATION WITH:} & \reportcollaboration \\
    \fi
    \textbf{Event Head:} & \eventhead \\
    \end{tabular}
    \end{flushleft}
    \vfill

\ifstandalone
    \end{titlepage}
\fi
}

% --- Conditional Front Matter (USE ARABIC EVERYWHERE) ---
\newcommand{\makeconditionalfrontmatter}{%
\ifstandalone
    % Only show these in individual reports
    \pagenumbering{arabic}
    \tableofcontents
    \listoffigures
    \clearpage
\else
    % In Annual Report mode: Do not show TOC, LOF, or LOT.
    % Pagination remains continuous Arabic from the master doc.
\fi
}

% --- Conditional Submission Block ---
\newcommand{\makesubmissionblock}{%
\ifstandalone
\vfill
\begin{flushleft}
    \large
    \textbf{REPORT SUBMITTED BY:} \\
    \vspace{0.5cm}
    \reportauthor \\
    AISSMS IOIT ACM Student Chapter \\
    \vspace{2cm}
    \rule{8cm}{0.4pt} \\
    (Signature) \\
    \vspace{1cm}
    \textbf{Date:} \submissiondate
\end{flushleft}
\fi
}

    \begin{document}
\fi

%===============================================================================
%   METADATA
%===============================================================================
% This section must be filled by the report author.
% These values are read by both Standalone and Annual Report modes.

\renewcommand{\reporttopic}{[Event Name/Title]}
\renewcommand{\reportdate}{[DD/MM/YYYY]}
\renewcommand{\reportvenue}{[Event Venue]}
\renewcommand{\reportorganizer}{AISSMS IOIT ACM Student Chapter}
\renewcommand{\eventhead}{[Event Head Name]}
\renewcommand{\reportauthor}{[Report Author Name]}
\renewcommand{\submissiondate}{[DD/MM/YYYY]}
\renewcommand{\reportcollaboration}{[Collaborating Organization / None]}

%===============================================================================
%   REPORT FRONT MATTER
%===============================================================================
% These macros are defined in common/event-macros.tex
% \makecustomtitle handles the title page (resets page in standalone, continuous in annual)
% \makeconditionalfrontmatter handles TOC/LOF/LOT (suppressed in annual mode)
\makecustomtitle
\makeconditionalfrontmatter

%===============================================================================
%   MAIN CONTENT
%===============================================================================

\section{Abstract}
[Provide a comprehensive 150-250 word summary of the event. Include the hosting organization, collaborating entities, event title, date, core concepts covered, target audience, total attendance, and a brief statement on the overall impact and success of the event.]

\section{Introduction}
[Provide background information leading to the conceptualization of this event. Explain the motivation for organizing it, the target audience, and the technological or academic gap it aimed to bridge. State how the event aligns with the goals of the ACM Student Chapter.]

\section{Event Objective}
[Detail the primary and secondary objectives of the event. What were the students expected to learn or achieve? Mention any practical skills, theoretical knowledge, or professional exposure the event aimed to deliver.]

\section{Event Details and Structure}
[Describe the overall structure of the event. Was it a single-day workshop, a multi-day competition, or a seminar? Mention the format (e.g., lecture, hands-on, interactive zones) and the general logistical setup managed by the organizing team.]

%-------------------------------------------------------------------------------
% DYNAMIC SECTION: RESOURCE PERSON DETAILS
% (Comment out if not applicable)
%-------------------------------------------------------------------------------
\section{Resource Person Details}
\begin{table}[h]
    \centering
    \begin{tabularx}{\textwidth}{@{}lX@{}}
        \hline
        \textbf{Name} & [Full Name] \\
        \hline
        \textbf{Designation} & [Professional Designation] \\
        \hline
        \textbf{Organization} & [Company / Institution Name] \\
        \hline
        \textbf{Email ID} & [Email Address] \\
        \hline
        \textbf{Expertise} & [Key Areas of Expertise] \\
        \hline
    \end{tabularx}
	\caption{Details of Resource Person}
\end{table}

%-------------------------------------------------------------------------------
% DYNAMIC SECTION: COMPETITION WINNERS
% (Comment out if not applicable)
%-------------------------------------------------------------------------------
\section{Competition Winners}
\begin{table}[h]
    \centering
	\begin{tabularx}{\textwidth}{@{}cXX@{}}
        \hline
        \textbf{Position} & \textbf{Name/Team Name} & \textbf{Department/Year} \\
        \hline
        1st Place & [Winner Name] & [Dept/Year] \\
        \hline
        2nd Place & [Winner Name] & [Dept/Year] \\
        \hline
        3rd Place & [Winner Name] & [Dept/Year] \\
        \hline
    \end{tabularx}
	\caption{Competition Winners}
\end{table}

\section{Event Execution and Activities}
[Provide a detailed, chronological account of how the event unfolded. Describe the specific topics covered by the speaker, the tools demonstrated, the nature of the hands-on activities, or the progression of the competition. Describe any distinct segments, breakout sessions, or interactive zones.]

\section{Participation and Engagement}
[Describe the demographics of the participants and their level of engagement. Mention specific moments of high interaction, such as Q\&A sessions, active participation in activities, or feedback received during the event.]

\begin{table}[h]
    \centering
	\begin{tabularx}{\textwidth}{@{}lc@{}}
        \toprule
        \textbf{Category} & \textbf{Count} \\
        \midrule
        Total Registrations & [00] \\
        Total Attendees & [00] \\
        \bottomrule
    \end{tabularx}
	\caption{Participation Statistics}
\end{table}

\section{Outcomes and Impact}
[Summarize the direct results of the event. Detail the specific technical and soft skills the participants gained. Provide any notable qualitative feedback or testimonials. Explain how the event contributed to the participants' academic or professional journey.]

\section{Glimpses from the Event}
\begin{figure}[h]
    \centering
    \includegraphics[width=0.5\textwidth]{1.jpeg}
    \caption{[Brief professional caption describing the action or subject in the first image]}
    \label{fig:session}
\end{figure}

\begin{figure}[h]
    \centering
    \includegraphics[width=0.5\textwidth]{2.jpeg}
    \caption{[Brief professional caption describing the interaction or audience in the second image]}
    \label{fig:interaction}
\end{figure}

\section{Conclusion}
[Provide a concluding paragraph summarizing the overall success of the event in meeting its objectives. Mention any recommendations or future steps, such as follow-up events, advanced workshops, or continuing collaborations.]

\section*{Acknowledgements}
We would like to express our sincere gratitude to our Faculty Coordinator, Dr. Meenakshi Thalor, for their continuous guidance and support. We extend our heartfelt thanks to the Principal, Dr. P. B. Mane, and the AISSMS IOIT administration for providing the necessary resources and platform to host this event. Special thanks go to the venue staff for their prompt logistical assistance. Finally, we deeply appreciate the Event Head, \eventhead, and all the organizing team members present at the venue whose dedicated efforts and coordination ensured the smooth and successful execution of the event.

%===============================================================================
%   SUBMISSION BLOCK
%===============================================================================
% Suppressed automatically in Annual Report mode via event-macros.tex
\makesubmissionblock

%===============================================================================
%   CONTEXT-AWARE CLOSER
%===============================================================================
\ifdefined\annualreportmode
    \clearpage
\else
    \end{document}
\fi

