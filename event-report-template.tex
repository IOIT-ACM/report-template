%%%%%%%%%%%%%%%%%%%%%%%%%%%%%%%%%%%%%%%%%%%%%%%%%%%%%%%%%%%%%%%%%%%%%%%%%%%%%%%%
%
%   IOIT ACM Event Report LaTeX Template
%   Version 2.0.0
%
%   Professional template for AISSMS IOIT ACM Student Chapter event reports.
%   Supports both standalone and annual report compilation modes.
%
%%%%%%%%%%%%%%%%%%%%%%%%%%%%%%%%%%%%%%%%%%%%%%%%%%%%%%%%%%%%%%%%%%%%%%%%%%%%%%%%
%
%   INSTRUCTION HEADER — READ BEFORE EDITING
%
%%%%%%%%%%%%%%%%%%%%%%%%%%%%%%%%%%%%%%%%%%%%%%%%%%%%%%%%%%%%%%%%%%%%%%%%%%%%%%%%
%
%   DOCUMENT STATE DETECTION
%
%   This document exists in exactly one of two states:
%
%   STATE 1 — TEMPLATE MODE
%   Identified by: Placeholder text in [brackets], instructional comments,
%   example table data, generic \newcommand values (DD/MM/YYYY, Event Name/Title)
%   Action: Generate complete event report following Generation Protocol
%
%   STATE 2 — COMPLETED REPORT MODE
%   Identified by: Real event content, populated tables with actual data,
%   specific event details in abstract/introduction, removal of placeholders
%   Action: Apply only explicitly requested updates following Update Protocol
%
%   DEFAULT RULE: If uncertain, treat as Template Mode.
%
%%%%%%%%%%%%%%%%%%%%%%%%%%%%%%%%%%%%%%%%%%%%%%%%%%%%%%%%%%%%%%%%%%%%%%%%%%%%%%%%
%
%   NON-NEGOTIABLE RULES
%
%   The following elements are institutionally mandated and MUST NEVER be modified:
%   - Document class and package declarations
%   - Custom command definitions structure
%   - Section order and required sections
%   - This instruction header
%
%   Violation of these rules invalidates the report for institutional submission.
%
%%%%%%%%%%%%%%%%%%%%%%%%%%%%%%%%%%%%%%%%%%%%%%%%%%%%%%%%%%%%%%%%%%%%%%%%%%%%%%%%
%
%   GENERATION PROTOCOL (Template Mode Only)
%
%%%%%%%%%%%%%%%%%%%%%%%%%%%%%%%%%%%%%%%%%%%%%%%%%%%%%%%%%%%%%%%%%%%%%%%%%%%%%%%%
%
%   RULE 1: DATA VERIFICATION REQUIREMENT
%
%   Do not generate without complete user-provided information.
%   If data is missing, output ONLY a categorized list of missing items.
%   No greetings, explanations, or commentary—just the list.
%
%   RULE 2: REQUIRED DATA CHECKLIST
%
%   Universal (All Events):
%   - Event name, date (DD/MM/YYYY), venue
%   - Event head name, report author name, submission date
%   - Registration and attendance numbers
%   - Event objectives, expected outcomes, detailed schedule
%   - Description of all activities conducted
%
%   Workshop/Seminar Specific:
%   - Resource person: Full name, designation, organization, email
%   - Resource person: Expertise area, background, topics covered
%
%   Competition Specific:
%   - Winner details: Names/team names (1st, 2nd, 3rd)
%   - Winner details: Department, year, prize information
%   - Judging criteria, problem statement/format
%
%   RULE 3: GENERATION STANDARDS
%
%   Structure:
%   - Preserve all LaTeX structure elements
%   - Maintain prescribed section order
%   - Remove all [bracketed placeholders]
%   - Keep ONLY this instruction header and Update Protocol
%
%   Content:
%   - Replace \newcommand definitions with actual data
%   - Write professional, formal prose
%   - Use sequential image filenames: 1.jpeg, 2.jpeg, 3.jpeg
%   - NO EMOJIS anywhere
%   - Include/comment out sections based on event type
%   - Document must be immediately compilable and submission-ready
%
%%%%%%%%%%%%%%%%%%%%%%%%%%%%%%%%%%%%%%%%%%%%%%%%%%%%%%%%%%%%%%%%%%%%%%%%%%%%%%%%
%
%   UPDATE PROTOCOL (Completed Report Mode Only)
%
%%%%%%%%%%%%%%%%%%%%%%%%%%%%%%%%%%%%%%%%%%%%%%%%%%%%%%%%%%%%%%%%%%%%%%%%%%%%%%%%
%
%   RULE 1: MINIMAL MODIFICATION PRINCIPLE
%
%   Apply surgical updates only. Modify exclusively requested elements.
%   Preserve all other content exactly as written.
%
%   RULE 2: UPDATE SCOPE IDENTIFICATION
%
%   Examples:
%   - "Change event date to 15/03/2024" → Modify \newcommand{\reportdate} only
%   - "Add second prize winner" → Add single row to winners table only
%   - "Update resource person email" → Modify email cell only
%   - "Expand outcomes section" → Modify Outcomes section only
%
%   RULE 3: PRESERVATION REQUIREMENTS
%
%   Never modify:
%   - LaTeX structure, packages, commands (per Non-Negotiable Rules)
%   - Sections not mentioned in update request
%   - Existing writing style, tone, or voice
%   - This instruction header
%
%   RULE 4: EXECUTION STANDARDS
%
%   Process:
%   a. Identify exact location of required change
%   b. Apply modification with minimal disruption
%   c. Maintain stylistic consistency
%   d. Verify document remains compilable
%   e. Ensure no emojis introduced
%   f. Return complete document with this header preserved
%
%%%%%%%%%%%%%%%%%%%%%%%%%%%%%%%%%%%%%%%%%%%%%%%%%%%%%%%%%%%%%%%%%%%%%%%%%%%%%%%%

\documentclass[11pt, a4paper]{article}

%===============================================================================
%   PREAMBLE AND MACROS
%===============================================================================

%%%%%%%%%%%%%%%%%%%%%%%%%%%%%%%%%%%%%%%%%%%%%%%%%%%%%%%%%%%%%%%%%%%%%%%%%%%%%%%%
%
%   COMMON PREAMBLE FOR IOIT ACM EVENT REPORTS
%   This file contains all shared package declarations and configurations
%   Used by both standalone event reports and the annual report
%
%%%%%%%%%%%%%%%%%%%%%%%%%%%%%%%%%%%%%%%%%%%%%%%%%%%%%%%%%%%%%%%%%%%%%%%%%%%%%%%%

% --- Font and Encoding ---
\usepackage[utf8]{inputenc}
\usepackage[T1]{fontenc}

% --- Page Layout and Geometry ---
\usepackage[left=2cm, right=2cm, top=3.5cm, bottom=2.5cm, headheight=75pt]{geometry}

% --- Essential Packages ---
\usepackage{graphicx}
\usepackage{fancyhdr}
\usepackage{lastpage}
\usepackage{hyperref}
\usepackage{xcolor}

% --- Content and Formatting Packages ---
\usepackage{titlesec}
\usepackage{enumitem}
\usepackage{booktabs}
\usepackage{caption}
\usepackage{subcaption}
\usepackage{amsmath, amssymb}
\usepackage{url}
\usepackage{longtable}
\usepackage{tabularx}

% --- Hyperlink Setup ---
\hypersetup{
    colorlinks=true,
    linkcolor=blue,
    filecolor=magenta,
    urlcolor=cyan,
    pdftitle={ACM Event Report},
    pdfauthor={AISSMS IOIT ACM Student Chapter},
    pdfsubject={Event Report}
}

% --- Header and Footer Configuration ---
\pagestyle{fancy}
\fancyhf{}

% Header Definition
\fancyhead[L]{\raisebox{0.3cm}{\includegraphics[height=1.5cm]{acm-logo.png}}}
\fancyhead[C]{\raisebox{0.2cm}{\includegraphics[height=2cm]{college-logo.png}}}
\renewcommand{\headrulewidth}{0.4pt}

% Footer Definition
\fancyfoot[C]{\thepage}
\renewcommand{\footrulewidth}{0pt}

% no header style
\fancypagestyle{noheader}{
\fancyhf{}
\fancyfoot[C]{\thepage}
\renewcommand{\headrulewidth}{0pt}
\renewcommand{\footrulewidth}{0pt}
}

%%%%%%%%%%%%%%%%%%%%%%%%%%%%%%%%%%%%%%%%%%%%%%%%%%%%%%%%%%%%%%%%%%%%%%%%%%%%%%%%
%
%   EVENT REPORT MACROS
%   Updated for Arabic Pagination and Conditional Front Matter
%
%%%%%%%%%%%%%%%%%%%%%%%%%%%%%%%%%%%%%%%%%%%%%%%%%%%%%%%%%%%%%%%%%%%%%%%%%%%%%%%%

% --- Mode Detection ---
\newif\ifstandalone
\ifdefined\annualreportmode
    \standalonefalse
\else
    \standalonetrue
\fi

% --- Metadata Placeholder Definitions ---
% These are default values. Reports will override these using \renewcommand.
\renewcommand{\reporttopic}{Event Name/Title Goes Here}
\renewcommand{\reportdate}{DD/MM/YYYY}
\renewcommand{\reportvenue}{AISSMS Institute of Information Technology}
\renewcommand{\reportorganizer}{AISSMS IOIT ACM Student Chapter}
\renewcommand{\eventhead}{Name of Event head}
\renewcommand{\reportauthor}{Report Author Name}
\renewcommand{\submissiondate}{DD/MM/YYYY}
\renewcommand{\reportcollaboration}{} % Add if needed

% --- Section Reset Logic for Annual Report ---
\newcounter{annualtempsection}
\newcounter{eventcount}

\newcommand{\includeevent}[1]{%
    \stepcounter{eventcount}%
    \setcounter{annualtempsection}{\value{section}}
    \setcounter{section}{0}
    \renewcommand{\theHsection}{event.\theeventcount.\arabic{section}}
    \input{#1}
    \setcounter{section}{\value{annualtempsection}}
    \renewcommand{\theHsection}{\arabic{section}}
}

% --- Custom Title Page Command ---
\newcommand{\makecustomtitle}{%
\ifstandalone
    \begin{titlepage}
    \thispagestyle{empty}
\else
    \clearpage
    \thispagestyle{fancy} % Standard header for annual report pages
\fi

    \centering
    \noindent
    \makebox[0.32\textwidth][l]{\includegraphics[height=2.4cm]{acm-logo.png}}%
    \makebox[0.36\textwidth][c]{\includegraphics[height=2.3cm]{college-logo.png}}%
    \makebox[0.32\textwidth][r]{}
    \vspace{2.5cm}

    {\Large \textbf{AISSMS IOIT ACM Student Chapter Event Report}}\\[0.5cm]
    {\Huge \bfseries \reporttopic}\\
    \vfill

    \begin{flushleft}
    \large
    \begin{tabular}{@{}l@{\quad}l}
    \textbf{EVENT DATE:} & \reportdate \\
    \textbf{VENUE:} & \reportvenue \\
    \textbf{ORGANIZED BY:} & \reportorganizer \\
    \ifx\reportcollaboration\empty\else
    \textbf{IN COLLABORATION WITH:} & \reportcollaboration \\
    \fi
    \textbf{Event Head:} & \eventhead \\
    \end{tabular}
    \end{flushleft}
    \vfill

\ifstandalone
    \end{titlepage}
\fi
}

% --- Conditional Front Matter (USE ARABIC EVERYWHERE) ---
\newcommand{\makeconditionalfrontmatter}{%
\ifstandalone
    % Only show these in individual reports
    \pagenumbering{arabic}
    \tableofcontents
    \listoffigures
    \clearpage
\else
    % In Annual Report mode: Do not show TOC, LOF, or LOT.
    % Pagination remains continuous Arabic from the master doc.
\fi
}

% --- Conditional Submission Block ---
\newcommand{\makesubmissionblock}{%
\ifstandalone
\vfill
\begin{flushleft}
    \large
    \textbf{REPORT SUBMITTED BY:} \\
    \vspace{0.5cm}
    \reportauthor \\
    AISSMS IOIT ACM Student Chapter \\
    \vspace{2cm}
    \rule{8cm}{0.4pt} \\
    (Signature) \\
    \vspace{1cm}
    \textbf{Date:} \submissiondate
\end{flushleft}
\fi
}


%===============================================================================
%   EVENT METADATA (CUSTOMIZE THESE)
%===============================================================================

\newcommand{\reporttopic}{Event Name/Title Goes Here}
\newcommand{\reportdate}{DD/MM/YYYY}
\newcommand{\reportvenue}{AISSMS Institute of Information Technology}
\newcommand{\reportorganizer}{AISSMS IOIT ACM Student Chapter}
\newcommand{\eventhead}{Name of Event head}
\newcommand{\reportcollaboration}{Collaboration Partner (if any)}
\newcommand{\reportauthor}{Report Author Name}
\newcommand{\submissiondate}{DD/MM/YYYY}

%===============================================================================
%   DOCUMENT BODY
%===============================================================================

\begin{document}

\makecustomtitle
\makeconditionalfrontmatter

%===============================================================================
%   MAIN CONTENT
%===============================================================================

%-------------------------------------------------------------------------------
\section{Abstract}
% PURPOSE: This section provides a concise summary of the entire event (150-250 words).
% CONTENT TO INCLUDE:
% - Brief overview of the event and its purpose
% - Event type (workshop, competition, seminar, hackathon, etc.)
% - Main objectives of organizing this event
% - Key activities conducted
% - Number of participants/attendees
% - Primary outcomes or key highlights
% Keep it brief and self-contained - readers should understand the essence of
% the event without reading the full report.

[Write your abstract here - a concise summary of the event, its objectives, activities conducted, participation, and key outcomes in 150-250 words]

%-------------------------------------------------------------------------------
\section{Introduction}
% PURPOSE: Set the context and background for the event.
% CONTENT TO INCLUDE:
% - Background about why this event was organized
% - Relevance to ACM Student Chapter's mission/goals
% - Target audience (students, professionals, specific departments, etc.)
% - Current trends or needs in the domain that this event addresses
% - Brief overview of what the report will cover
% - Any historical context (if this is a recurring event)
% This section should gradually lead the reader from general context to
% the specific event.

[Write your introduction here - provide background context, explain why this event was organized, its relevance to students/community, and give an overview of the report]

%-------------------------------------------------------------------------------
\section{Event Objective}
% PURPOSE: Clearly state what the event aimed to achieve.
% CONTENT TO INCLUDE:
% - Primary goal(s) of organizing the event
% - Learning outcomes expected for participants
% - Skills or knowledge to be developed
% - Community building or networking objectives
% - Expected impact on students/participants
% - Specific deliverables or outcomes planned
% Be specific and measurable where possible.

[State the main goals and specific objectives of the event - what did you aim to achieve? What learning outcomes were expected? What impact did you plan to create?]

%-------------------------------------------------------------------------------
\section{Event Details and Structure}
% PURPOSE: Describe the organization and structure of the event.
% CONTENT TO INCLUDE:
% - Event format (workshop, competition, seminar, panel discussion, etc.)
% - Duration and schedule (single-day, multi-day, time slots)
% - Registration process and participant selection (if applicable)
% - Event agenda or timeline
% - Activities/sessions conducted
% - Tools, platforms, or technologies used (online/offline)
% - Team involved in organizing (roles and responsibilities)
% This section should give readers a clear understanding of how the
% event was structured and executed.

[Describe the format, structure, schedule, and organization of the event. Include the agenda, activities conducted, and how the event was managed]

\subsection{Event Agenda}
% PURPOSE: Provide detailed schedule of the event.
% CONTENT TO INCLUDE:
% - Session-wise breakdown with timings
% - Topics covered in each session
% - Speaker/facilitator for each session
% - Break times and networking sessions
% Use a table or itemized list for clarity.

[Provide the detailed agenda/schedule of the event with timings and session details]

\begin{table}[h]
    \centering
    \begin{tabularx}{\textwidth}{@{}lX@{}}
        \toprule
        \textbf{Time} & \textbf{Session/Activity} \\
        \midrule
        10:00 AM - 10:30 AM & Registration \& Welcome \\
        10:30 AM - 11:30 AM & [Session 1 Topic] \\
        11:30 AM - 12:30 PM & [Session 2 Topic] \\
        12:30 PM - 01:30 PM & Lunch Break \\
        01:30 PM - 03:00 PM & [Session 3 Topic] \\
        03:00 PM - 04:00 PM & [Final Activity/Closing] \\
        \bottomrule
    \end{tabularx}
	\caption{Event Schedule}
	\label{tab:schedule}
\end{table}

%-------------------------------------------------------------------------------
% DYNAMIC SECTION 1: RESOURCE PERSON DETAILS
% INSTRUCTIONS: Keep this section for WORKSHOPS, SEMINARS, GUEST LECTURES, and
% similar events where external experts/speakers are invited.
% COMMENT OUT this entire section for events like COMPETITIONS, HACKATHONS
% (unless they have guest speakers/judges), or INTERNAL ACTIVITIES.
%-------------------------------------------------------------------------------
\section{Resource Person Details}
% PURPOSE: Provide information about the expert(s) who conducted the event.
% CONTENT TO INCLUDE:
% - Complete details as per the table format
% - Name, designation, organization/company
% - Email ID and contact information
% - Area of expertise relevant to the event
% - Brief background or credentials (can be added in paragraph form below table)
% - Topics/sessions they handled
% This section is crucial for workshops, seminars, and guest lectures.
% COMMENT OUT THIS SECTION if your event type doesn't have resource persons.

[Fill in the details of the resource person(s) who conducted the workshop/seminar. Use the table format provided below]

\begin{table}[h]
    \centering
    \begin{tabularx}{\textwidth}{@{}lX@{}}
        \hline
        \textbf{Name} & [Full Name of Resource Person] \\
        \hline
        \textbf{Designation} & [Job Title/Position] \\
        \hline
        \textbf{Organization/Company} & [Organization/Company Name] \\
        \hline
        \textbf{Email ID} & [email@example.com] \\
        \hline
        \textbf{Area of Expertise} & [e.g., Machine Learning, Web Development, Cloud Computing, Cybersecurity, etc.] \\
        \hline
    \end{tabularx}
	\caption{Details of Industry Expert / Resource Person}
	\label{tab:resource_person}
\end{table}

% If multiple resource persons, duplicate the table:
% \begin{table}[h]
%     [Same table structure as above]
% \end{table}

%-------------------------------------------------------------------------------
% DYNAMIC SECTION 2: COMPETITION WINNERS
% INSTRUCTIONS: Keep this section for COMPETITIONS, HACKATHONS, CODING CONTESTS,
% and similar competitive events.
% COMMENT OUT this entire section for events like WORKSHOPS, SEMINARS, GUEST
% LECTURES, or NON-COMPETITIVE ACTIVITIES.
%-------------------------------------------------------------------------------
\section{Competition Winners}
% PURPOSE: Recognize and document the winners of the competition.
% CONTENT TO INCLUDE:
% - Winner details with positions (1st, 2nd, 3rd place, special mentions)
% - Participant names (individual or team names)
% - Department/Year (if applicable)
% - Brief description of winning entry/performance
% - Prizes awarded
% This section is essential for all competitive events.
% COMMENT OUT THIS SECTION if your event is not a competition.

[Document the competition winners with their positions, names, and details. Add photos if available]

\subsection{Winner List}
% List the winners in order of their rankings

\begin{table}[h]
    \centering
	\begin{tabularx}{\textwidth}{@{}cXX@{}}
        \hline
        \textbf{Position} & \textbf{Name/Team Name} & \textbf{Department/Year} \\
        \hline
        1st Place & [Winner Name/Team] & [Dept/Year] \\
        \hline
        2nd Place & [Winner Name/Team] & [Dept/Year] \\
        \hline
        3rd Place & [Winner Name/Team] & [Dept/Year] \\
        \hline
        % Add more rows for special mentions, consolation prizes, etc.
    \end{tabularx}
	\caption{Competition Winners}
	\label{tab:winners}
\end{table}

%-------------------------------------------------------------------------------
\section{Event Execution and Activities}
% PURPOSE: Describe what actually happened during the event.
% CONTENT TO INCLUDE:
% - Detailed description of sessions/activities conducted
% - Key topics covered or skills taught
% - Hands-on activities, exercises, or tasks
% - Participant engagement and interaction
% - Demonstrations, presentations, or practical sessions
% - Q&A sessions, discussions, or collaborative activities
% This section should give readers a vivid picture of the event experience.

[Describe the actual execution of the event - what activities were conducted, what topics were covered, how participants engaged, and any notable moments]

%-------------------------------------------------------------------------------
\section{Participation and Engagement}
% PURPOSE: Document participant statistics and engagement levels.
% CONTENT TO INCLUDE:
% - Total number of participants/registrations
% - Department-wise or year-wise breakdown
% - Attendance statistics
% - Participant feedback or response during the event
% - Level of engagement and interaction
% - Any participation certificates or recognition given
% Use tables or charts to present statistics clearly.

[Provide detailed information about participation numbers, demographics, and engagement levels during the event]

\begin{table}[h]
    \centering
	\begin{tabularx}{\textwidth}{@{}lc@{}}
        \toprule
        \textbf{Category} & \textbf{Count} \\
        \midrule
        Total Registrations & [Number] \\
        Total Attendees & [Number] \\
        \bottomrule
    \end{tabularx}
	\caption{Participation Statistics}
	\label{tab:participation}
\end{table}

%-------------------------------------------------------------------------------
\section{Outcomes and Impact}
% PURPOSE: Present and analyze the results and impact of the event.
% CONTENT TO INCLUDE:
% - Whether event objectives were achieved
% - Learning outcomes for participants
% - Skills developed or knowledge gained
% - Feedback received from participants
% - Unexpected positive outcomes or learnings
% - Success metrics (attendance vs. registration, engagement levels, etc.)
% This section should demonstrate the value and impact of organizing the event.

[Describe the outcomes and impact of the event. How successful was it? What did participants learn or gain? Was the event effective in achieving its objectives?]

\subsection{Participant Feedback}
% PURPOSE: Include qualitative feedback from attendees.
% CONTENT TO INCLUDE:
% - Summary of feedback collected (surveys, forms, verbal feedback)
% - Positive comments and testimonials
% - Areas of improvement suggested by participants
% - Overall satisfaction ratings
% - Quotes from participants (if available)

[Summarize the feedback received from participants. Include both quantitative ratings and qualitative comments]

\begin{itemize}
    \item Overall satisfaction rating: [X/5 or percentage]
    \item Positive feedback highlights: [Key positive comments]
    \item Areas for improvement: [Suggestions from participants]
    \item Notable testimonials: [Direct quotes if available]
\end{itemize}

\subsection{Knowledge and Skills Gained}
% PURPOSE: Document the educational value of the event.
% CONTENT TO INCLUDE:
% - Technical skills acquired or improved
% - Soft skills developed (teamwork, communication, problem-solving)
% - Practical exposure to industry tools/technologies
% - Career or academic insights gained
% - Networking opportunities created

[Describe the knowledge, skills, and experience gained by participants through this event. What specific competencies were developed?]

\begin{itemize}
    \item Technical skill 1: [Description]
    \item Technical skill 2: [Description]
    \item Soft skill 1: [Description]
    \item Practical exposure: [Description]
    \item Career insights: [Description]
\end{itemize}

\subsection{Long-term Impact}
% PURPOSE: Discuss the broader or lasting impact of the event.
% CONTENT TO INCLUDE:
% - How this event contributes to students' career readiness
% - Community building within the ACM chapter
% - Industry connections established
% - Follow-up activities or continued learning opportunities

[Discuss the long-term or broader impact of this event on participants, the ACM chapter, and the student community]

%-------------------------------------------------------------------------------
\section{Glimpses from the Event}
% PURPOSE: Provide visual documentation of the event.
% CONTENT TO INCLUDE:
% - Photos from the event (venue, participants, activities, presentations)
% - Screenshots (for online events or demonstrations)
% - Resource person delivering sessions
% - Participant engagement moments
% - Group photos
% Each figure should have a clear, descriptive caption explaining what it shows.
% Reference figures in the text using \ref{fig:label}
% Image files should be named: 1.jpeg, 2.jpeg, 3.jpeg, etc.

[Include photographs and other visual documentation from the event. Each figure should have a descriptive caption. Replace image placeholders with actual filenames: 1.jpeg, 2.jpeg, 3.jpeg]

\begin{figure}[h]
    \centering
    \includegraphics[width=0.8\textwidth]{1.jpeg}
    \caption{[Caption describing the image - e.g., "Resource person delivering the workshop session on Machine Learning"]}
    \label{fig:session}
\end{figure}

\begin{figure}[h]
    \centering
    \begin{subfigure}{0.4\textwidth}
        \includegraphics[width=\linewidth]{2.jpeg}
        \caption{[Caption for first image]}
        \label{fig:glimpse1}
    \end{subfigure}
    \hfill
    \begin{subfigure}{0.4\textwidth}
        \includegraphics[width=\linewidth]{3.jpeg}
        \caption{[Caption for second image]}
        \label{fig:glimpse2}
    \end{subfigure}
    \caption{[Overall caption - e.g., "Participants actively engaged in hands-on activities"]}
    \label{fig:activities}
\end{figure}

%-------------------------------------------------------------------------------
\section{Conclusion}
% PURPOSE: Wrap up the report and provide recommendations.
% CONTENT TO INCLUDE:
% - Summary of the event's success
% - Restatement of how objectives were met
% - Overall impact on participants and ACM chapter
% - Potential improvements or enhancements
% - Suggestions for follow-up activities or next steps
% - Vision for continuing this event (if recurring)
% Keep this section concise and forward-looking.

[Summarize the event's success, restate achievement of objectives, and provide recommendations for future events]

%-------------------------------------------------------------------------------
\section*{Acknowledgements}
% PURPOSE: Recognize contributions and support.
% CONTENT TO INCLUDE:
% - Faculty coordinator (Dr. Meenakshi Thalor) and mentors
% - ACM chapter office bearers and organizing committee members
% - Resource persons/guest speakers/judges
% - Collaborating organizations, companies, or industry partners
% - Sponsors or funding sources (if applicable)
% - College administration and departments for support
% - Volunteers and supporting staff
% - Any other individuals or groups who contributed
% Be specific and sincere in expressing gratitude.

[Acknowledge all the individuals, faculty members, organizations, and institutions that supported and contributed to making this event successful. Express gratitude for their guidance, resources, and collaboration]

We would like to express our sincere gratitude to:
\begin{itemize}
    \item Faculty Coordinator Dr. Meenakshi Thalor, for guidance and support
    \item Resource Person Name, for conducting an excellent workshop/seminar/session
    \item College Department, for providing venue and resources
    \item Sponsoring Organization, for their generous support
    \item ACM Student Chapter office bearers and organizing team members
    \item All participants for their enthusiastic engagement
\end{itemize}

%===============================================================================
%   SUBMISSION BLOCK
%===============================================================================

\makesubmissionblock

\end{document}
%===============================================================================
%   END OF DOCUMENT
%===============================================================================
