%%%%%%%%%%%%%%%%%%%%%%%%%%%%%%%%%%%%%%%%%%%%%%%%%%%%%%%%%%%%%%%%%%%%%%%%%%%%%%%%
%
%   IOIT ACM ANNUAL REPORT TEMPLATE
%   Version 2.1.0
%
%   This template aggregates all individual event reports into a single
%   annual report with continuous pagination and global indices.
%
%%%%%%%%%%%%%%%%%%%%%%%%%%%%%%%%%%%%%%%%%%%%%%%%%%%%%%%%%%%%%%%%%%%%%%%%%%%%%%%%

\documentclass[11pt, a4paper]{article}

%%%%%%%%%%%%%%%%%%%%%%%%%%%%%%%%%%%%%%%%%%%%%%%%%%%%%%%%%%%%%%%%%%%%%%%%%%%%%%%%
%
%   COMMON PREAMBLE FOR IOIT ACM EVENT REPORTS
%   This file contains all shared package declarations and configurations
%   Used by both standalone event reports and the annual report
%
%%%%%%%%%%%%%%%%%%%%%%%%%%%%%%%%%%%%%%%%%%%%%%%%%%%%%%%%%%%%%%%%%%%%%%%%%%%%%%%%

% --- Font and Encoding ---
\usepackage[utf8]{inputenc}
\usepackage[T1]{fontenc}

% --- Page Layout and Geometry ---
\usepackage[left=2cm, right=2cm, top=3.5cm, bottom=2.5cm, headheight=75pt]{geometry}

% --- Essential Packages ---
\usepackage{graphicx}
\usepackage{fancyhdr}
\usepackage{lastpage}
\usepackage{hyperref}
\usepackage{xcolor}

% --- Content and Formatting Packages ---
\usepackage{titlesec}
\usepackage{enumitem}
\usepackage{booktabs}
\usepackage{caption}
\usepackage{subcaption}
\usepackage{amsmath, amssymb}
\usepackage{url}
\usepackage{longtable}
\usepackage{tabularx}

% --- Hyperlink Setup ---
\hypersetup{
    colorlinks=true,
    linkcolor=blue,
    filecolor=magenta,
    urlcolor=cyan,
    pdftitle={ACM Event Report},
    pdfauthor={AISSMS IOIT ACM Student Chapter},
    pdfsubject={Event Report}
}

% --- Header and Footer Configuration ---
\pagestyle{fancy}
\fancyhf{}

% Header Definition
\fancyhead[L]{\raisebox{0.3cm}{\includegraphics[height=1.5cm]{acm-logo.png}}}
\fancyhead[C]{\raisebox{0.2cm}{\includegraphics[height=2cm]{college-logo.png}}}
\renewcommand{\headrulewidth}{0.4pt}

% Footer Definition
\fancyfoot[C]{\thepage}
\renewcommand{\footrulewidth}{0pt}

% no header style
\fancypagestyle{noheader}{
\fancyhf{}
\fancyfoot[C]{\thepage}
\renewcommand{\headrulewidth}{0pt}
\renewcommand{\footrulewidth}{0pt}
}

\usepackage{fontawesome5}

%===============================================================================
%   ANNUAL REPORT MODE DECLARATION
%===============================================================================
% This flag tells event reports they are being compiled in embedded mode.
% It triggers the suppression of local TOCs and enables section resetting.
\newcommand{\annualreportmode}{}
\def\annualreportmode{}

%%%%%%%%%%%%%%%%%%%%%%%%%%%%%%%%%%%%%%%%%%%%%%%%%%%%%%%%%%%%%%%%%%%%%%%%%%%%%%%%
%
%   EVENT REPORT MACROS
%   Updated for Arabic Pagination and Conditional Front Matter
%
%%%%%%%%%%%%%%%%%%%%%%%%%%%%%%%%%%%%%%%%%%%%%%%%%%%%%%%%%%%%%%%%%%%%%%%%%%%%%%%%

% --- Mode Detection ---
\newif\ifstandalone
\ifdefined\annualreportmode
    \standalonefalse
\else
    \standalonetrue
\fi

% --- Metadata Placeholder Definitions ---
% These are default values. Reports will override these using \renewcommand.
\renewcommand{\reporttopic}{Event Name/Title Goes Here}
\renewcommand{\reportdate}{DD/MM/YYYY}
\renewcommand{\reportvenue}{AISSMS Institute of Information Technology}
\renewcommand{\reportorganizer}{AISSMS IOIT ACM Student Chapter}
\renewcommand{\eventhead}{Name of Event head}
\renewcommand{\reportauthor}{Report Author Name}
\renewcommand{\submissiondate}{DD/MM/YYYY}
\renewcommand{\reportcollaboration}{} % Add if needed

% --- Section Reset Logic for Annual Report ---
\newcounter{annualtempsection}
\newcounter{eventcount}

\newcommand{\includeevent}[1]{%
    \stepcounter{eventcount}%
    \setcounter{annualtempsection}{\value{section}}
    \setcounter{section}{0}
    \renewcommand{\theHsection}{event.\theeventcount.\arabic{section}}
    \input{#1}
    \setcounter{section}{\value{annualtempsection}}
    \renewcommand{\theHsection}{\arabic{section}}
}

% --- Custom Title Page Command ---
\newcommand{\makecustomtitle}{%
\ifstandalone
    \begin{titlepage}
    \thispagestyle{empty}
\else
    \clearpage
    \thispagestyle{fancy} % Standard header for annual report pages
\fi

    \centering
    \noindent
    \makebox[0.32\textwidth][l]{\includegraphics[height=2.4cm]{acm-logo.png}}%
    \makebox[0.36\textwidth][c]{\includegraphics[height=2.3cm]{college-logo.png}}%
    \makebox[0.32\textwidth][r]{}
    \vspace{2.5cm}

    {\Large \textbf{AISSMS IOIT ACM Student Chapter Event Report}}\\[0.5cm]
    {\Huge \bfseries \reporttopic}\\
    \vfill

    \begin{flushleft}
    \large
    \begin{tabular}{@{}l@{\quad}l}
    \textbf{EVENT DATE:} & \reportdate \\
    \textbf{VENUE:} & \reportvenue \\
    \textbf{ORGANIZED BY:} & \reportorganizer \\
    \ifx\reportcollaboration\empty\else
    \textbf{IN COLLABORATION WITH:} & \reportcollaboration \\
    \fi
    \textbf{Event Head:} & \eventhead \\
    \end{tabular}
    \end{flushleft}
    \vfill

\ifstandalone
    \end{titlepage}
\fi
}

% --- Conditional Front Matter (USE ARABIC EVERYWHERE) ---
\newcommand{\makeconditionalfrontmatter}{%
\ifstandalone
    % Only show these in individual reports
    \pagenumbering{arabic}
    \tableofcontents
    \listoffigures
    \clearpage
\else
    % In Annual Report mode: Do not show TOC, LOF, or LOT.
    % Pagination remains continuous Arabic from the master doc.
\fi
}

% --- Conditional Submission Block ---
\newcommand{\makesubmissionblock}{%
\ifstandalone
\vfill
\begin{flushleft}
    \large
    \textbf{REPORT SUBMITTED BY:} \\
    \vspace{0.5cm}
    \reportauthor \\
    AISSMS IOIT ACM Student Chapter \\
    \vspace{2cm}
    \rule{8cm}{0.4pt} \\
    (Signature) \\
    \vspace{1cm}
    \textbf{Date:} \submissiondate
\end{flushleft}
\fi
}


\begin{document}

%===============================================================================
%   TITLE Page
%===============================================================================

\begin{titlepage}
\centering
\thispagestyle{empty}

\includegraphics[height=2.8cm]{college-logo.png}

\vspace{2.5cm}

% Main title
{\LARGE \textcolor{blue}{\textbf{AISSMS IOIT ACM}}}\\[0.3cm]
{\LARGE \textcolor{blue}{\textbf{Student Chapter}}}

\vspace{2cm}

% ACM Logo
\includegraphics[height=4.5cm]{acm-logo.png}

\vspace{2cm}

% Academic Report text
{\LARGE \textcolor{blue}{\textbf{Academic Report (2025-26)}}}

\vfill

\end{titlepage}

%===============================================================================
%   TITLE Page End
%===============================================================================

\pagenumbering{arabic}
\newcommand{\makeannualtitle}{
\begin{titlepage}
\centering
\thispagestyle{empty}

\noindent
\makebox[0.32\textwidth][l]{%
    \includegraphics[height=2.4cm]{acm-logo.png}
}%
\makebox[0.36\textwidth][c]{%
    \includegraphics[height=2.3cm]{college-logo.png}
}%
\makebox[0.32\textwidth][r]{%
}
\vspace{2.5cm}

{\Huge \bfseries \annualreporttitle}\\[0.5cm]
{\Large \academicyear}\\
\vfill

\begin{flushleft}
\large
\begin{tabular}{@{}l@{\quad}l}
\textbf{PREPARED BY:} & AISSMS IOIT ACM Student Chapter \\
\textbf{INSTITUTION:} & AISSMS Institute of Information Technology \\
\textbf{SUBMISSION DATE:} & 12 Dec 2025 \\
\end{tabular}
\end{flushleft}

\vfill
\end{titlepage}
}


\newpage

% Executive Summary
\section*{Executive Summary}
\addcontentsline{toc}{section}{Executive Summary}

This annual report presents a comprehensive overview of all events and activities conducted by the AISSMS IOIT ACM Student Chapter during the academic year 2024-2025. The chapter successfully organized [X] events with a total participation of [Y] students across various domains including workshops, seminars, competitions, and technical sessions.

%===============================================================================
%   CHAPTER/SECTION: INTRODUCTION
%===============================================================================

\section{Introduction}

The AISSMS IOIT ACM Student Chapter is dedicated to fostering technical excellence, innovation, and professional development among students. Throughout the academic year 2024-2025, the chapter organized a diverse range of events aimed at enhancing technical skills, promoting industry interaction, and building a vibrant learning community.

\title{What is ACM and Objectives of ACM}

\noindent
\textbf{ACM (Association for Computing Machinery)} is a global scientific and educational organization dedicated to advancing computing as a science and profession. Founded in 1947, it's the world's largest computing society, bringing together educators, researchers, and professionals in the field of computer science and information technology.

\section*{Objectives of ACM:}

\subsection*{1. Bringing Together Computing Professionals}
ACM serves as a platform that unites computing educators, researchers, and professionals from across the globe. This collective approach helps in sharing knowledge and expertise within the computing community.

\subsection*{2. Promoting a Scientific Mindset}
The organization is committed to fostering scientific thinking and approaches in computing. ACM chapters operate exclusively for educational and scientific purposes, emphasizing rigorous methodologies and evidence-based practices.

\subsection*{3. Supporting Special Interest Groups}
ACM facilitates greater interest in computing and its applications through specialized communities focused on particular areas of computing technology and practice.

\subsection*{4. Building Professional Networks}
ACM provides a means of communication between persons having an interest in computing. This networking aspect allows professionals to connect, collaborate, and share ideas.

\subsection*{5. Making an Impact}
The organization aims to increase knowledge of and interest in various aspects of computing, including:
\begin{itemize}
    \item Science and theoretical foundations
    \item Design methodologies
    \item Development practices
    \item Construction techniques
    \item Programming languages
    \item Management approaches
    \item Applications of modern computing
\end{itemize}

\subsection*{6. Resource Sharing}
ACM inspires dialogue and shares resources among its members, providing access to cutting-edge research, best practices, and educational materials.

\subsection*{7. Digital Library}
The ACM Digital Library (DL) is one of the most comprehensive collections of computing literature, providing members with access to journals, conference proceedings, technical magazines, and more.

\subsection*{8. Addressing Field Challenges}
ACM works to address critical challenges in the computing field, including ethical issues, technological advancements, and educational needs.

\section*{ACM Student Chapters}
A significant part of ACM's structure is its student chapters, which:
\begin{itemize}
    \item Operate on college and university campuses
    \item Provide students with opportunities to learn beyond the classroom
    \item Enable networking with industry professionals
    \item Offer leadership experience
    \item Focus exclusively on educational and scientific purposes
    \item Create platforms for students with computing interests to connect and collaborate
\end{itemize}

%===============================================================================
%   CHAPTER/SECTION: COMMUNITY BUILDING AND PROFESSIONAL DEVELOPMENT
%===============================================================================

\section{Community Building and Professional Development}

Beyond conducting individual technical events, the AISSMS IOIT ACM Student Chapter plays a vital role in cultivating a strong professional ecosystem that emphasizes collaboration, communication, and long-term skill development among students.

\subsection{Building Professional and Peer Networks}

One of the primary objectives of the chapter is to build meaningful professional and peer networks within and beyond the institute. Through workshops, speaker sessions, collaborative events, and chapter meetings, students gain opportunities to interact with:
\begin{itemize}
    \item Fellow computing enthusiasts across academic years
    \item Faculty members and academic mentors
    \item Industry professionals and alumni
    \item Speakers from research and development backgrounds
\end{itemize}

These interactions help students broaden their perspectives, gain career guidance, and form long-lasting professional relationships.

\subsection{Enhancing Technical Communication Skills}

The chapter places strong emphasis on improving students' technical communication abilities. Members are encouraged to:
\begin{itemize}
    \item Present technical topics during seminars and peer-learning sessions
    \item Participate in discussions, panel sessions, and Q\&A forums
    \item Document event outcomes, technical learnings, and project insights
    \item Engage in collaborative problem-solving and knowledge sharing
\end{itemize}

Such activities enable students to articulate complex technical concepts clearly and confidently—an essential skill for academic, research, and industry environments.

\subsection{Promoting Open Source Culture}

Promoting awareness and participation in open-source development is a key initiative of the ACM Student Chapter. Students are encouraged to:
\begin{itemize}
    \item Explore open-source technologies and platforms
    \item Contribute to public repositories and community-driven projects
    \item Understand collaborative development workflows and best practices
    \item Learn version control, documentation standards, and issue tracking
\end{itemize}

This exposure helps students develop real-world development experience while fostering a culture of collaboration and shared learning.

\subsection{Encouraging Competitive Participation}

The chapter actively motivates students to participate in technical competitions, hackathons, and problem-solving contests. These competitive platforms help students:
\begin{itemize}
    \item Apply theoretical knowledge to practical challenges
    \item Develop analytical and critical thinking skills
    \item Work effectively under time constraints
    \item Gain exposure to national and international-level competitions
\end{itemize}

Such participation enhances confidence, technical depth, and readiness for future academic and professional challenges.

\subsection{Collaboration with Student-Led Initiatives}

The ACM Student Chapter maintains a collaborative relationship with other prominent student-driven initiatives within the institute, including:

\subsubsection{IOIT TENET}
IOIT TENET represents the institute’s annual techfest:
\begin{itemize}
    \item Joint technical events and innovation-driven activities
    \item Cross-domain learning and interdisciplinary exposure
    \item Leadership development through coordinated planning and execution
\end{itemize}

\subsubsection{IOIT Model United Nations (IOIT MUN)}
While IOIT MUN primarily focuses on diplomacy, leadership, and communication, collaboration with ACM supports:
\begin{itemize}
    \item Development of organizational and coordination skills
    \item Exposure to structured discussions and formal communication
    \item Broader student engagement across technical and non-technical domains
\end{itemize}

These collaborations contribute to the holistic development of students by balancing technical excellence with leadership, communication, and organizational skills.

\subsection{Student Leadership and Team Development}

The success of the ACM Student Chapter is driven by a dedicated team of student office bearers and volunteers. Through active involvement, students gain:
\begin{itemize}
    \item Leadership and responsibility management experience
    \item Event planning and execution skills
    \item Team coordination and decision-making abilities
    \item Exposure to professional conduct and accountability
\end{itemize}

This structured involvement prepares students for leadership roles in both academic and professional settings.


%===============================================================================
%   CHAPTER/SECTION: DETAILED EVENT REPORTS
%===============================================================================

\section{Detailed Event Reports}

The following sections contain detailed reports for each event conducted during the academic year.

\newpage

\includeevent{2025/events/github/index}
\includeevent{2025/events/buildonsui/index}

%===============================================================================
%   CHAPTER/SECTION: OVERALL IMPACT AND ACHIEVEMENTS
%===============================================================================

\section{Overall Impact and Achievements}

Throughout the academic year, the ACM Student Chapter successfully:

\begin{itemize}
    \item Conducted [X] technical events with [Y] total participants
    \item Provided hands-on training in [list key technical domains]
    \item Facilitated industry interactions through [number] expert sessions
    \item Awarded [number] students in various competitions
    \item Built a strong technical community with enhanced collaboration
\end{itemize}

%===============================================================================
%   CHAPTER/SECTION: FUTURE PLANS AND RECOMMENDATIONS
%===============================================================================

\section{Future Plans and Recommendations}

Based on this year's experiences, the following plans and recommendations are proposed for the next academic year:

\begin{enumerate}
    \item Expand the scope of technical workshops to include emerging technologies
    \item Strengthen industry partnerships for better placement opportunities
    \item Increase student participation through targeted outreach programs
    \item Introduce more competitive events to foster innovation
    \item Develop a mentorship program connecting seniors with juniors
\end{enumerate}

%===============================================================================
%   CHAPTER/SECTION: ACKNOWLEDGEMENTS
%===============================================================================

\section*{Acknowledgements}
\addcontentsline{toc}{section}{Acknowledgements}

We would like to express our sincere gratitude to:

\begin{itemize}
    \item Faculty Coordinator Dr. Meenakshi Thalor, for continuous guidance and support
    \item College Principal and Management, for providing resources and institutional support
    \item All resource persons, industry experts, and guest speakers who shared their knowledge
    \item All ACM Student Chapter office bearers and volunteers for their dedication
    \item All participating students who made these events successful
\end{itemize}

%===============================================================================
%   SUBMISSION BLOCK
%===============================================================================

\clearpage
\vfill
\begin{flushleft}
    \large
    \textbf{ANNUAL REPORT PREPARED BY:} \\
    \vspace{0.5cm}
    AISSMS IOIT ACM Student Chapter \\
    \vspace{2cm}
    \rule{8cm}{0.4pt} \\
    (Signature) \\
    \vspace{1cm}
    \textbf{Date:} 12 Dec 2025
\end{flushleft}

\end{document}
%===============================================================================
%   END OF ANNUAL REPORT
%===============================================================================
